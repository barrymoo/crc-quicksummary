\documentclass[landscape,a0b,final]{a0poster}

% Useful Packages
\usepackage{stix}
\usepackage{epsfig}
\usepackage{multicol}
\usepackage{pstricks,pst-grad}
\usepackage{minted}
\usepackage{hyperref}
\usepackage[T1]{fontenc}

%%%%%%%%%%%%%%%%%%%%%%%%%%%%%%%%%%%%%%%%%%%
% Definition of some variables and colors
%%%%%%%%%%%%%%%%%%%%%%%%%%%%%%%%%%%%%%%%%%%
\setlength{\columnsep}{3cm}
\setlength{\columnseprule}{2mm}
\setlength{\parindent}{0.0cm}

% Useful Commands
\newcommand{\code}[1]{\mintinline{bash}{#1}}
\newcommand{\myurl}[1]{\href{#1}{\color{blue}{#1}}}
\newcommand{\pygment}[3]{\inputminted[bgcolor=lightgray,linenos,fontsize=#1]{#2}{#3}}
\newcommand{\ig}[2]{\includegraphics[width=#1\linewidth]{#2}}
\newcommand{\mysection}[1]{
    \begin{center}
        \pbox{0.8\columnwidth}{}{linewidth=3mm,framearc=0.3,linecolor=black,fillstyle=gradient,gradangle=0,gradbegin=white,gradend=white,gradmidpoint=1.0,framesep=0.5em}{
            \begin{center}
                \Large\color{black}{\bf{#1}}
            \end{center}
        }
    \end{center}
    \vspace{0.25cm}
}

\newcommand{\background}[3]{
  \newrgbcolor{cgradbegin}{#1}
  \newrgbcolor{cgradend}{#2}
  \psframe[fillstyle=gradient,gradend=cgradend,
  gradbegin=cgradbegin,gradmidpoint=#3](0.,0.)(1.\textwidth,-1.\textheight)
}

\newenvironment{poster}{
  \begin{center}
  \begin{minipage}[c]{0.981\textwidth}
}{
  \end{minipage} 
  \end{center}
}

\newenvironment{pcolumn}[1]{
  \begin{minipage}{#1\textwidth}
  \begin{center}
}{
  \end{center}
  \end{minipage}
}

\newcommand{\pbox}[4]{
\psshadowbox[#3]{
\begin{minipage}[t][#2][t]{#1}
#4
\end{minipage}
}}

\newcommand{\myfig}[3][0]{
\begin{center}
  \vspace{0.25cm}
  \includegraphics[width=#3\hsize,angle=#1]{#2}
  \nobreak\medskip
\end{center}}

\setcounter{figure}{1}
\newcommand{\mycaption}[1]{
  \vspace{0.25cm}
  \begin{quote}
    {{\sc Figure} \arabic{figure}: #1}
  \end{quote}
  \vspace{0.25cm}
  \stepcounter{figure}
}

%%%%%%%%%%%%%%%%%%%%%%%%%%%%%%%%%%%%%%%%%%%%%%%%%%%%%%%%%%%%%%%%%%%%%%
%%% Begin of Document
%%%%%%%%%%%%%%%%%%%%%%%%%%%%%%%%%%%%%%%%%%%%%%%%%%%%%%%%%%%%%%%%%%%%%%

\begin{document}

\newrgbcolor{pittblue}{.11 .16 .34}
\newrgbcolor{pittgold}{.80 .72 .49}
\newrgbcolor{white}{1. 1. 1.}
\newrgbcolor{black}{0. 0. 0.}

\background{1. 1. 1.}{1. 1. 1.}{0.5}

\vspace*{0.25cm} %Controls white space on top of poster

\begin{poster}

%%%%%%%%%%%%%%%%%%%%%
%%% Header
%%%%%%%%%%%%%%%%%%%%%
\begin{center}
\begin{pcolumn}{0.98}

\pbox{0.98\textwidth}{}{linewidth=1.75mm,framearc=0.3,linecolor=black,fillstyle=gradient,gradangle=0,gradbegin=white,gradend=white,gradmidpoint=1.0,framesep=1em}{
    \begin{minipage}[c][7cm][c]{0.98\textwidth}
    \begin{center}
        \color{black}{
            \hspace{5cm}
            {\sc \Huge Quick Summary}\\%[5mm] 
            \hspace{5cm}
            {\sc \Huge Center for Research Computing\\[5mm]
            \hspace{5cm}
            {\sc \Huge University of Pittsburgh}}
        }
    \end{center}
    \end{minipage}
}
\end{pcolumn}
\end{center}

\vspace*{0.05cm} %Spacing between banner and text

%%%%%%%%%%%%%%%%%%%%%
%%% Content
%%%%%%%%%%%%%%%%%%%%%

%%% Begin of Multicols-Enviroment
\begin{multicols}{3}

    \mysection{Useful Links}
    \Large
    \begin{center}
        \begin{itemize}
            \item H2P Documentation: \\
                \myurl{https://crc.pitt.edu/h2p}
            \item HTC Documentation: \\
                \myurl{https://crc.pitt.edu/htc}
            \item Submit a Ticket: \\
                \myurl{https://crc.pitt.edu/tickets}
            \item CRC Status: \\
                \myurl{https://crc.pitt.edu/status}
            \item Proposals, Renewals, and Allocations: \\
                \myurl{https://crc.pitt.edu/apply}
            \item Hardware Description: \\
                \myurl{https://crc.pitt.edu/Resource-Description-Proposal-Language}
            \item Submit Citations: \\
                \myurl{https://crc.pitt.edu/doi}
            \item Submit Feedback: \\
                \myurl{https://crc.pitt.edu/contact}
        \end{itemize}
    \end{center}

    \vspace{0.1cm}
    \mysection{Access Environments}
    \Large
    \begin{center}
        \begin{itemize}
            \item \textbf{Connect to the VPN when off-campus or on Pitt WiFi!}
                \begin{itemize}
                    \item Details: \myurl{https://crc.pitt.edu/h2p\#Off-campus-access}
                \end{itemize}
            \item Login Nodes: \code{h2p.crc.pitt.edu} or \code{htc.crc.pitt.edu}
            \item Jupyter Notebooks: \myurl{https://hub.crc.pitt.edu}
            \item Visualization: \myurl{https://viz.crc.pitt.edu}
            \item Open OnDemand: \myurl{https://ondemand.htc.crc.pitt.edu}
            \item Galaxy: \myurl{https://crc.pitt.edu/galaxy}
            \item CLC Genomics: \myurl{https://crc.pitt.edu/clcbioserver}
        \end{itemize}
    \end{center}

    \vspace{0.1cm}
    \mysection{Storage}
    \Large    
    \begin{center}
        \begin{itemize}
            \item Mount points and quotas:
            \begin{itemize}
                \item Isilon (75G per user) \code{/ihome}
                \item ZFS (5TB per group) \code{/zfs1} or \code{/zfs2}
                \item BGFS (5TB per group) \code{/bgfs}
            \end{itemize}
            \item Policies:
            \begin{itemize}
                \item Your home directory will be located in \code{/ihome} and is backed up
                \item BGFS and ZFS are \textbf{not} backed up
                \item Each group can have up to 5TB on BGFS or ZFS for free
                \item To get access to BGFS or ZFS submit a ticket
            \end{itemize}
        \end{itemize}
    \end{center}

    \vspace{0.1cm}
    \mysection{Useful Commands}
    \Large    
    \begin{center}
        \begin{itemize}
            \item Connect via secure shell \code{ssh <username>@h2p.crc.pitt.edu}
            \item Transferring data:
            \begin{itemize}
                \item \code{scp <file> <username>@h2p.crc.pitt.edu:~} \\
                    \code{# local to remote}
                \item \code{scp <username>@h2p.crc.pitt.edu:~/<file> .} \\
                    \code{# remote to local}
            \end{itemize}
            \item Check your filesystem quotas: \code{crc-quota.py}
            \item Check your service units quotas: \code{crc-usage.pl}
        \end{itemize}
    \end{center}

    \vspace{0.1cm}
    \mysection{Software}
    \Large    
    \begin{center}
        \begin{itemize}
            \item Search all available software: \code{module spider} \\
                \code{# Use '/' for search, 'q' quits}
            \item Load a module: \code{module load <module>}
            \item List loaded modules: \code{module list}
            \item Purge your modules: \code{module purge}
        \end{itemize}
    \end{center}

    \vspace{0.1cm}
    \mysection{Getting Help}
    \Large    
    \begin{center}
        \begin{itemize}
            \item When submitting a support ticket:
                \begin{itemize}
                    \item Use a descriptive and specific title
                        \begin{itemize}
                            \item Bad: Command doesn't work
                            \item Good: Gaussian is failing with <abbreviated error here>
                        \end{itemize}
                    \item Provide enough information for us to reproduce your problem, e.g.
                        \begin{itemize}
                            \item The input and output file with exact paths
                        \end{itemize}
                \end{itemize}
            \item CRC Consultants with Expertise
                \begin{itemize}
                    \item Kim Wong: Molecular Dynamics | Agent-based Modeling
                    \item Fangping Mu: Bioinformatics | Computational Biology/Genomics
                    \item Barry Moore II: Computational Chemistry | Deep Learning
                    \item Shervin Sammak: Computational Fluid Dynamics
                    \item Leonardo Bernasconi: Computational Chemistry
                    \item Firas Shomali: High Performance Computing
                \end{itemize}
            \item Comments and suggestions for improvement can be submitted at
                \myurl{https://crc.pitt.edu/contact}
        \end{itemize}
    \end{center}

    \vspace{0.1cm}
    \mysection{Slurm Jobs}
    \Large    
    \begin{center}
        \begin{itemize}
            \item Show your jobs only: \code{crc-squeue.py}
            \item Cancel a job: \code{crc-scancel.py <jobid>}
            \item Submit an interactive job: \code{crc-interactive.py -s}
            \item Submit a batch job: \code{sbatch <file>}
            \item Browse clusters and partitions: \code{crc-squeue.py}
            \item Add this to the end of your batch scripts: \code{crc-job-stats.py}
            \item Example job scripts, browse \code{/ihome/crc/how_to_run}
            \item Description of \code{sbatch} arguments:
            \begin{itemize}
                \item \code{--job-name}: This name shows up in \code{crc-squeue.py}
                \item \code{--output}: Standard input/output will be written to this file
                \item \code{--time}: How long your job will run for \\
                    (format: \code{[days-]hours:minutes:seconds}
                \item \code{--nodes}: How many nodes your job needs
                \item \code{--ntasks-per-node}: How many MPI processes your job
                    needs per node, pro tip: if you don't know what this means
                    it should be equal to 1
                \item \code{--cpus-per-task}: Either the number of cores or OpenMP threads
                \item \code{--cluster}: Run on this cluster, try \code{crc-sinfo.py} to see clusters
                \item \code{--partition}: Run on this partition, try \code{crc-sinfo.py} to see partitions
                \item \code{--mail-user}: Set to your Pitt email address to get updates (optional), pair with \code{--mail-type=END,FAIL}
            \end{itemize}
            \item Reasons from \code{crc-squeue.py} (don't submit a ticket unless specified)
                \begin{itemize}
                    \item \code{Priority}: You are waiting in line
                    \item \code{MaxCPUPerAccount}: You are waiting in line because your group is running on too many cores
                    \item \code{MaxMemoryPerAccount}: You are waiting in line because your group is using too much memory
                    \item \code{GrpTresRunMinsLimit}: Your group is out of service units. Submit a ticket!
                \end{itemize}
        \end{itemize}
    \end{center}

    \vspace{0.1cm}
    \mysection{Notes}
    \Large    
    \begin{center}
        \begin{itemize}
            \item Is something not working? \myurl{https://crc.pitt.edu/tickets}
            \item Suggestion or improvement for this document: \\
                \myurl{https://github.com/barrymoo/crc-quicksummary/issues/new}
        \end{itemize}
    \end{center}

\end{multicols}

\end{poster}
\end{document}
